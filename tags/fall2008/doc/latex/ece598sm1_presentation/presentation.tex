\documentclass{beamer}
\usepackage[latin1]{inputenc}

% Add the compsoc option for Computer Society conferences.
%
% If IEEEtran.cls has not been installed into the LaTeX system files,
% manually specify the path to it like:
% \documentclass[conference]{../sty/IEEEtran}

% Some very useful LaTeX packages include:
% (uncomment the ones you want to load)

\newif\ifans\anstrue
% This is a package for drawing figures
% it is a part of standard latex 2e distribution
\usepackage{tikz}
\usetikzlibrary{shapes}
\usetikzlibrary{automata}

\usepackage[usenames,dvipsnames]{pstricks}
\usepackage{epsfig}
%\usepackage{pst-grad} % For gradients
%\usepackage{pst-plot} % For axes

\usepackage{palatino}
\RequirePackage{ifthen}
\usepackage{latexsym}
\RequirePackage{amsmath}
\RequirePackage{amsthm}
\RequirePackage{amssymb}
\RequirePackage{xspace}
\RequirePackage{graphics}
\usepackage{xcolor}
%\usepackage{fullpage}
%\usepackage{schemabloc}

\RequirePackage{textcomp}
\usepackage{keyval}
%\usepackage{listings}
%\usepackage{xspace}
\usepackage{mathrsfs}
%\usepackage{textcomp}
%\usepackage{graphicx}
\usepackage{paralist}
\usepackage{amsmath,amssymb,url,listings,mathrsfs}
%\usepackage{pvs}
%\usepackage{supertabular,alltt,latexsym}
%\usepackage{multicol,multirow,epsfig}
%\usepackage[dvips, usenames]{color}
\usepackage{framed}
\usepackage{lipsum}
%\usepackage[dvipsnames]{color}



\definecolor{reddish}{rgb}{1,.8,0.8}
\definecolor{blueish}{rgb}{0.8,.8,1}
\definecolor{greenish}{rgb}{.8,1,0.8}
\definecolor{yellowish}{rgb}{1,1,.20}

% *** MISC UTILITY PACKAGES ***
%'
%\usepackage{ifpdf}
% Heiko Oberdiek's ifpdf.sty is very useful if you need conditional
% compilation based on whether the output is pdf or dvi.
% usage:
% \ifpdf
%   % pdf code
% \else
%   % dvi code
% \fi
% The latest version of ifpdf.sty can be obtained from:
% http://www.ctan.org/tex-archive/macros/latex/contrib/oberdiek/
% Also, note that IEEEtran.cls V1.7 and later provides a builtin
% \ifCLASSINFOpdf conditional that works the same way.
% When switching from latex to pdflatex and vice-versa, the compiler may
% have to be run twice to clear warning/error messages.






% *** CITATION PACKAGES ***
%
\usepackage{cite}
% cite.sty was written by Donald Arseneau
% V1.6 and later of IEEEtran pre-defines the format of the cite.sty package
% \cite{} output to follow that of IEEE. Loading the cite package will
% result in citation numbers being automatically sorted and properly
% "compressed/ranged". e.g., [1], [9], [2], [7], [5], [6] without using
% cite.sty will become [1], [2], [5]--[7], [9] using cite.sty. cite.sty's
% \cite will automatically add leading space, if needed. Use cite.sty's
% noadjust option (cite.sty V3.8 and later) if you want to turn this off.
% cite.sty is already installed on most LaTeX systems. Be sure and use
% version 4.0 (2003-05-27) and later if using hyperref.sty. cite.sty does
% not currently provide for hyperlinked citations.
% The latest version can be obtained at:
% http://www.ctan.org/tex-archive/macros/latex/contrib/cite/
% The documentation is contained in the cite.sty file itself.










% *** MATH PACKAGES ***
%
%\usepackage[cmex10]{amsmath}
% A popular package from the American Mathematical Society that provides
% many useful and powerful commands for dealing with mathematics. If using
% it, be sure to load this package with the cmex10 option to ensure that
% only type 1 fonts will utilized at all point sizes. Without this option,
% it is possible that some math symbols, particularly those within
% footnotes, will be rendered in bitmap form which will result in a
% document that can not be IEEE Xplore compliant!
%
% Also, note that the amsmath package sets \interdisplaylinepenalty to 10000
% thus preventing page breaks from occurring within multiline equations. Use:
%\interdisplaylinepenalty=2500
% after loading amsmath to restore such page breaks as IEEEtran.cls normally
% does. amsmath.sty is already installed on most LaTeX systems. The latest
% version and documentation can be obtained at:
% http://www.ctan.org/tex-archive/macros/latex/required/amslatex/math/





% *** SPECIALIZED LIST PACKAGES ***
%
%\usepackage{algorithmic}
% algorithmic.sty was written by Peter Williams and Rogerio Brito.
% This package provides an algorithmic environment fo describing algorithms.
% You can use the algorithmic environment in-text or within a figure
% environment to provide for a floating algorithm. Do NOT use the algorithm
% floating environment provided by algorithm.sty (by the same authors) or
% algorithm2e.sty (by Christophe Fiorio) as IEEE does not use dedicated
% algorithm float types and packages that provide these will not provide
% correct IEEE style captions. The latest version and documentation of
% algorithmic.sty can be obtained at:
% http://www.ctan.org/tex-archive/macros/latex/contrib/algorithms/
% There is also a support site at:
% http://algorithms.berlios.de/index.html
% Also of interest may be the (relatively newer and more customizable)
% algorithmicx.sty package by Szasz Janos:
% http://www.ctan.org/tex-archive/macros/latex/contrib/algorithmicx/




% *** ALIGNMENT PACKAGES ***
%
%\usepackage{array}
% Frank Mittelbach's and David Carlisle's array.sty patches and improves
% the standard LaTeX2e array and tabular environments to provide better
% appearance and additional user controls. As the default LaTeX2e table
% generation code is lacking to the point of almost being broken with
% respect to the quality of the end results, all users are strongly
% advised to use an enhanced (at the very least that provided by array.sty)
% set of table tools. array.sty is already installed on most systems. The
% latest version and documentation can be obtained at:
% http://www.ctan.org/tex-archive/macros/latex/required/tools/


%\usepackage{mdwmath}
%\usepackage{mdwtab}
% Also highly recommended is Mark Wooding's extremely powerful MDW tools,
% especially mdwmath.sty and mdwtab.sty which are used to format equations
% and tables, respectively. The MDWtools set is already installed on most
% LaTeX systems. The lastest version and documentation is available at:
% http://www.ctan.org/tex-archive/macros/latex/contrib/mdwtools/


% IEEEtran contains the IEEEeqnarray family of commands that can be used to
% generate multiline equations as well as matrices, tables, etc., of high
% quality.


%\usepackage{eqparbox}
% Also of notable interest is Scott Pakin's eqparbox package for creating
% (automatically sized) equal width boxes - aka "natural width parboxes".
% Available at:
% http://www.ctan.org/tex-archive/macros/latex/contrib/eqparbox/





% *** SUBFIGURE PACKAGES ***
%\usepackage[tight,footnotesize]{subfigure}
% subfigure.sty was written by Steven Douglas Cochran. This package makes it
% easy to put subfigures in your figures. e.g., "Figure 1a and 1b". For IEEE
% work, it is a good idea to load it with the tight package option to reduce
% the amount of white space around the subfigures. subfigure.sty is already
% installed on most LaTeX systems. The latest version and documentation can
% be obtained at:
% http://www.ctan.org/tex-archive/obsolete/macros/latex/contrib/subfigure/
% subfigure.sty has been superceeded by subfig.sty.



%\usepackage[caption=false]{caption}
%\usepackage[font=footnotesize]{subfig}
% subfig.sty, also written by Steven Douglas Cochran, is the modern
% replacement for subfigure.sty. However, subfig.sty requires and
% automatically loads Axel Sommerfeldt's caption.sty which will override
% IEEEtran.cls handling of captions and this will result in nonIEEE style
% figure/table captions. To prevent this problem, be sure and preload
% caption.sty with its "caption=false" package option. This is will preserve
% IEEEtran.cls handing of captions. Version 1.3 (2005/06/28) and later 
% (recommended due to many improvements over 1.2) of subfig.sty supports
% the caption=false option directly:
%\usepackage[caption=false,font=footnotesize]{subfig}
%
% The latest version and documentation can be obtained at:
% http://www.ctan.org/tex-archive/macros/latex/contrib/subfig/
% The latest version and documentation of caption.sty can be obtained at:
% http://www.ctan.org/tex-archive/macros/latex/contrib/caption/




% *** FLOAT PACKAGES ***
%
%\usepackage{fixltx2e}
% fixltx2e, the successor to the earlier fix2col.sty, was written by
% Frank Mittelbach and David Carlisle. This package corrects a few problems
% in the LaTeX2e kernel, the most notable of which is that in current
% LaTeX2e releases, the ordering of single and double column floats is not
% guaranteed to be preserved. Thus, an unpatched LaTeX2e can allow a
% single column figure to be placed prior to an earlier double column
% figure. The latest version and documentation can be found at:
% http://www.ctan.org/tex-archive/macros/latex/base/



%\usepackage{stfloats}
% stfloats.sty was written by Sigitas Tolusis. This package gives LaTeX2e
% the ability to do double column floats at the bottom of the page as well
% as the top. (e.g., "\begin{figure*}[!b]" is not normally possible in
% LaTeX2e). It also provides a command:
%\fnbelowfloat
% to enable the placement of footnotes below bottom floats (the standard
% LaTeX2e kernel puts them above bottom floats). This is an invasive package
% which rewrites many portions of the LaTeX2e float routines. It may not work
% with other packages that modify the LaTeX2e float routines. The latest
% version and documentation can be obtained at:
% http://www.ctan.org/tex-archive/macros/latex/contrib/sttools/
% Documentation is contained in the stfloats.sty comments as well as in the
% presfull.pdf file. Do not use the stfloats baselinefloat ability as IEEE
% does not allow \baselineskip to stretch. Authors submitting work to the
% IEEE should note that IEEE rarely uses double column equations and
% that authors should try to avoid such use. Do not be tempted to use the
% cuted.sty or midfloat.sty packages (also by Sigitas Tolusis) as IEEE does
% not format its papers in such ways.





% *** PDF, URL AND HYPERLINK PACKAGES ***
%
%\usepackage{url}
% url.sty was written by Donald Arseneau. It provides better support for
% handling and breaking URLs. url.sty is already installed on most LaTeX
% systems. The latest version can be obtained at:
% http://www.ctan.org/tex-archive/macros/latex/contrib/misc/
% Read the url.sty source comments for usage information. Basically,
% \url{my_url_here}.





% *** Do not adjust lengths that control margins, column widths, etc. ***
% *** Do not use packages that alter fonts (such as pslatex).         ***
% There should be no need to do such things with IEEEtran.cls V1.6 and later.
% (Unless specifically asked to do so by the journal or conference you plan
% to submit to, of course. )






















\usetheme{Malmoe}
%\useoutertheme[footline=]{miniframes}
\title[Simplex Architecture Controlled Inverted Pendulum]{Stability Analysis Case Study:\\ Simplex Architecture Controlled Inverted Pendulum}
\author{Taylor Johnson}
\institute{University of Illinois at Urbana-Champaign\\ECE598SM1\\Fall 2008}
\date{December 4, 2008}

\setbeamertemplate{headline}
{%
	\begin{beamercolorbox}[right,rightskip=15pt]{pagenum}
		\vskip2pt(\insertframenumber/\inserttotalframenumber)\vskip2pt
	\end{beamercolorbox}%
}

\begin{document}
% marking changes

%\newcommand{\mitras}[1]{\textcolor{blue}{#1}}
\newcommand{\mitras}[1]{{#1}}
% for nomencl to produce page links
%\renewcommand*{\pagedeclaration}[1]{\unskip, \hyperpage{#1}}
%\renewcommand{\nomname}{List of Symbols and Functions}

\newcommand{\authcomment}[1]{\textbf{[[#1]]}}
\newcommand{\sayan}[1]{\textbf{[[#1]]}}


%FIXED SETS

%\newcommand{\Time}{{\sf T}}                     %time domain
\newcommand{\nnt}{{\sf T}^{\geq 0}}             %nonnegative time points
\newcommand{\post}{{\sf T}^{>0}}                %positive time points
\newcommand{\Variables}{{\sf V}}                %variables

\newcommand{\num}[1]{\relax\ifmmode \mathbb #1\else $\mathbb #1$\fi}
\newcommand{\nnnum}[1]{\relax\ifmmode 
  {\mathbb #1}_{\geq 0} \else ${\mathbb #1}_{\geq 0}$
  \fi}
\newcommand{\npnum}[1]{\relax\ifmmode 
  {\mathbb #1}_{\leq 0} \else ${\mathbb #1}_{\leq 0}$
  \fi}
\newcommand{\pnum}[1]{\relax\ifmmode 
  {\mathbb #1}_{> 0} \else ${\mathbb #1}_{> 0}$
  \fi}
\newcommand{\nnum}[1]{\relax\ifmmode 
  {\mathbb #1}_{< 0} \else ${\mathbb #1}_{< 0}$
  \fi}
\newcommand{\plnum}[1]{\relax\ifmmode 
  {\mathbb #1}_{+} \else ${\mathbb #1}_{+}$
  \fi}
\newcommand{\nenum}[1]{\relax\ifmmode 
  {\mathbb #1}_{-} \else ${\mathbb #1}_{-}$
  \fi}

\newcommand{\reals}{{\num R}}                    %reals
\newcommand{\nnreals}{{\nnnum R}}                    %nonnegative reals
\newcommand{\realsinfty}{{\num R} \cup \{\infty, -\infty\}}                    %nonnegative reals
\newcommand{\plreals}{{\plnum R}}                    %positive reals
\newcommand{\naturals}{{\num N}}                      %natural numbers
\newcommand{\integers}{{\num Z}}                      %integers
\newcommand{\rationals}{{\num Q}}                      %rationals
\newcommand{\nnrationals}{{\nnnum Q}}                   % nonnegative rationals
\newcommand{\Time}{{\num T}}  

% EXECUTIONS TRACES and FRAGS
\newcommand{\extb}[1]{\relax\ifmmode {\sf ExtBeh}_{#1} \else ${\sf ExtBeh}_{#1}$\fi} 
\newcommand{\tdists}[1]{\relax\ifmmode {\sf Tdists}_{#1} \else ${\sf Tdists}_{#1}$\fi} 

\newcommand{\exec}[1]{\relax\ifmmode {\sf Execs}_{#1} \else ${\sf Exec}_{#1}$\fi} 
\newcommand{\execf}[1]{\relax\ifmmode {\sf Execs}^*_{#1} \else ${\sf Exec}^*_{#1}$\fi} 
\newcommand{\execi}[1]{\relax\ifmmode {\sf Execs}^\omega_{#1} \else ${\sf Exec}^\omega_{#1}$\fi} 

\newcommand{\ctrace}[1]{\relax\ifmmode {\sf Ctraces}_{#1} \else ${\sf Ctraces}_{#1}$\fi} 

\newcommand{\trace}[1]{\relax\ifmmode {\sf Traces}_{#1} \else ${\sf Traces}_{#1}$\fi} 
\newcommand{\tracef}[1]{\relax\ifmmode {\sf Traces}^*_{#1} \else ${\sf Traces}^*_{#1}$\fi} 
\newcommand{\tracei}[1]{\relax\ifmmode {\sf Traces}^\omega_{#1} \else ${\sf Traces}^\omega_{#1}$\fi} 

\newcommand{\frag}[1]{\relax\ifmmode {\sf Frags}_{#1} \else ${\sf Frags}_{#1}$\fi} 
\newcommand{\fragf}[1]{\relax\ifmmode {\sf Frags}^*_{#1} \else ${\sf Frags}^*_{#1}$\fi} 
\newcommand{\fragi}[1]{\relax\ifmmode {\sf Frags}^\omega_{#1} \else ${\sf Frags}^\omega_{#1}$\fi} 

\newcommand{\reach}[1]{\relax\ifmmode {\sf Reach}_{#1} \else ${\sf Reach}_{#1}$\fi} 

\newcommand{\execs}{{\exec{}}}
\newcommand{\traces}{{\trace{}}}
\newcommand{\fragss}{{\frag{}}}

\newcommand{\fexecs}{{\execf{}}}
\newcommand{\ftraces}{{\tracef{}}}
\newcommand{\ffragss}{{\fragf{}}}

\newcommand{\iexecs}{{\execi{}}}
\newcommand{\itraces}{{\tracei{}}}
\newcommand{\ifragss}{{\fragi{}}}


\newenvironment{noqedproof}{\pf}{}
\newcommand{\pf}{\par\noindent{\bf Proof:}~}


% \newtheorem{theorem}{Theorem}[section]
% \newtheorem{lemma}[theorem]{Lemma}
% \newtheorem{corollary}[theorem]{Corollary}
 \newtheorem{claim}[theorem]{Claim}
\theoremstyle{definition}
% \newtheorem{definition}{Definition}[section]
 \renewcommand{\qed}{\hfill{\rule{2mm}{2mm}}\medskip}
 %\newenvironment{proof}{\pf}{\qed}
 \newtheorem{proposition}[theorem]{Proposition}
 \newtheorem{inv}[theorem]{Invariant}
 \newtheorem{remark}[theorem]{Remark}

 
 %\newenvironment{example}
 %{\refstepcounter{theorem} \vspace{2ex}\par\noindent
 %\textbf{Example}~\textbf{\thetheorem~}}{\qed}
 
\theoremstyle{remark}
 \newcounter{example}
% \newtheorem{example}{Example}[chapter]
% {\refstepcounter{theorem} \vspace{2ex}\par\noindent
% \textbf{Example}~\textbf{\theexampletheorem~}}{\qed}
 
 
 \def\examplenonum#1#2{
	\vspace{.15in}
	\noindent
	{\bf Example #1}{\em ~(continued)}{\bf .}
	{#2}{\qed}
	}
% 
% \renewcommand{\theequation}{\thesection.\arabic{equation}}
% \renewcommand{\thefigure}{\thesection.\arabic{figure}}
% \renewcommand{\thetable}{\thesection.\arabic{table}}
% \newcommand{\Section}[1]{\section{#1}%
%    \setcounter{equation}{0}\setcounter{figure}{0}\setcounter{table}{0}%
%    \setcounter{example}{0}}
% 
% \newenvironment{anotation}[1][Nancy]{\begin{quote}\small[[[#1]]]}{\normalsize\end{quote}}
% \newcommand{\ba}{\begin{anotation}}
% \newcommand{\ea}{\end{anotation}}
% 
% \newcommand{\bcb}{\chgbarbegin}
% \newcommand{\ecb}{\chgbarend}
% \chgbarwidth 1pt
% 
% \newcommand{\dec}{\ensuremath{{:}}}
% \newcommand{\eqdef}{\mathbin{::=}}
% \newcommand{\I}{{\ensuremath{\cap}}}



% OPERATIONS ON SETS, RELATIONS AND FUNCTIONS
\newcommand{\pow}[1]{{\bf P}(#1)} % powerset
\newcommand{\inverse}[1]{#1^{-1}}
\newcommand{\range}[1]{\ms{range{(#1)}}}
\newcommand{\domain}[1]{{\it dom}(#1)}
\newcommand{\type}[1]{\ms{type{(#1)}}}
\newcommand{\dtype}[1]{\ms{dtype{(#1)}}} % dynamic type
\newcommand{\restr}{\mathrel{\lceil}}
\newcommand{\proj}{\matrel{\lceil}}
\newcommand{\restrrange}{\mathrel{\downarrow}}
\newcommand{\point}[1]{\wp(#1)}     		%point trajectory



% HYBRID AUTOMATA
\def\A{{\cal A}} % HA
\def\B{{\cal B}} % HA
\def\C{{\cal C}} % HA
\def\D{{\cal D}} % set of discrete steps
\def\E{{\cal E}} % HA
\def\F{{\cal F}} % HA
\def\G{{\cal G}} % pieces of SHIOA
\def\H{{\cal H}} % HA
\def\I{{\cal I}} % environment sequence
\def\K{{\cal K}} % environment sequence
\def\M{{\cal M}} % Mode switching transitions
\def\O{{\cal O}} % outcome function
\def\P{{\cal P}} % set of modes
\def\Q{{\cal Q}} % set of modes
\def\R{{\cal R}} % relation
\def\S{{\cal S}} % set of trajectories
\def\T{{\cal T}} % set of trajectories
\def\V{{\cal V}} % Lyapunov function
\def\U{{\cal U}} % set of trajectories
\def\X{{\cal X}} % Lyapunov function
\def\Y{{\cal Y}} % set of trajectories
\def\KL{{\cal KL}}


% more special characters

\newcommand{\col}[1]{\relax\ifmmode \mathscr #1\else $\mathscr #1$\fi}

\def\statemodels{\col{S}}


% Names of actions, automata etc
\definecolor{HIOAcolor}{rgb}{0.776,0.22,0.07}
\newcommand{\HIOA}{\textcolor{HIOAcolor}{\tt HIOA\hspace{3pt}}}
\newcommand{\PVS}{\textcolor{HIOAcolor}{\tt PVS\hspace{3pt}}}
\newcommand{\PVSnogap}{\textcolor{HIOAcolor}{\tt PVS\hspace{1pt}}}
\newcommand{\HIOAbiggap}{\textcolor{HIOAcolor}{\tt HIOA\hspace{6pt}}}
\newcommand{\HIOAnogap}{\textcolor{HIOAcolor}{\tt HIOA}}
\newcommand{\anyrelation}{\lessgtr}

% Transformation for ADT
\newcommand{\SC}[2]{\relax\ifmmode {\tt Scount}(#1,#2) \else ${\tt Scount}(#1,#2)$\fi} 
\newcommand{\SCM}[2]{\relax\ifmmode {\tt Smin}(#1,#2) \else ${\tt Smin}(#1,#2)$\fi} 
\newcommand{\Aut}[1]{\relax\ifmmode {\tt Aut}(#1) \else ${\tt Aut}(#1)$\fi} 

\newcommand{\auto}[1]{{\operatorname{\mathsf{#1}}}}
\newcommand{\act}[1]{{\operatorname{\mathsf{#1}}}}
\newcommand{\smodel}[1]{{\operatorname{\mathsf{#1}}}}
\newcommand{\pvstheory}[1]{{\operatorname{\mathit{#1}}}}
%\newcommand{\auto}[1]{\relax\ifmmode \sf #1\else $\sf #1$\fi}
%\newcommand{\act}[1]{\relax\ifmmode \sf #1\else $\sf #1$\fi}


\newcommand{\Automaton}{{\bf automaton}}
\newcommand{\Asserts}{{\bf asserts}}
\newcommand{\Assumes}{{\bf assumes}}
\newcommand{\Backward}{{\bf backward}}
\newcommand{\By}{{\bf by}}
\newcommand{\Case}{{\bf case}}
\newcommand{\Choose}{{\bf  choose}}
\newcommand{\Components}{{\bf components}}
\newcommand{\Const}{{\bf const}}
\newcommand{\Converts}{{\bf converts}}
\newcommand{\Do}{{\bf do}}
\newcommand{\Eff}{{\bf eff}}
\newcommand{\Else}{{\bf else}}
\newcommand{\Elseif}{{\bf elseif}}
\newcommand{\Enumeration}{{\bf enumeration}}
\newcommand{\Ensuring}{{\bf ensuring}}
\newcommand{\Exempting}{{\bf exempting}}
\newcommand{\Fi}{{\bf fi}}
\newcommand{\For}{{\bf for}}
\newcommand{\Forward}{{\bf forward}}
\newcommand{\Freely}{{\bf freely}}
\newcommand{\From}{{\bf from}}
\newcommand{\Generated}{{\bf generated}}
\newcommand{\Local}{{\bf local}}
\newcommand{\Hidden}{{\bf hidden}}
\newcommand{\If}{{\bf if}}
\newcommand{\In}{{\bf in}}
\newcommand{\Implies}{{\bf implies}}
\newcommand{\Includes}{{\bf includes}}
\newcommand{\Introduces}{{\bf introduces}}
\newcommand{\Input}{{\bf input}}
\newcommand{\Kind}{{\bf kind}}
\newcommand{\Initially}{{\bf initially}}
\newcommand{\Internal}{{\bf internal}}
\newcommand{\Invariant}{{\bf invariant}}
\newcommand{\Od}{{\bf od}}
\newcommand{\Of}{{\bf of}}
\newcommand{\Output}{{\bf output}}
\newcommand{\Partitioned}{{\bf partitioned}}
\newcommand{\Pre}{{\bf pre}}
\newcommand{\Signature}{{\bf signature}}
\newcommand{\Simulation}{{\bf simulation}}
\newcommand{\Sort}{{\bf sort}}
\newcommand{\States}{{\bf states}}
\newcommand{\Tasks}{{\bf tasks}}
\newcommand{\Then}{{\bf then}}
\newcommand{\To}{{\bf to}}
\newcommand{\Trait}{{\bf trait}}
\newcommand{\Traits}{{\bf traits}}
\newcommand{\Transitions}{{\bf transitions}}
\newcommand{\Tuple}{{\bf tuple}}
\newcommand{\Type}{{\bf type}}
\newcommand{\Union}{{\bf union}}
\newcommand{\Uses}{{\bf uses}}
\newcommand{\Where}{{\bf where}}
\newcommand{\While}{{\bf while}}
\newcommand{\With}{{\bf with}}


% Spacing
\newcommand{\FFF}{\vspace{0.1in}}
\newcommand{\BBB}{\hspace{-0.1in}}


\newcommand{\deq}{\mathrel{\stackrel{\scriptscriptstyle\Delta}{=}}}


% systematic labels

\newcommand{\seclabel}[1]{\label{sec:#1}}
\newcommand{\secref}[1]{Section~\ref{sec:#1}}
\newcommand{\secreftwo}[2]{Sections~\ref{sec:#1}~and~\ref{sec:#2}}
\newcommand{\figlabel}[1]{\label{fig:#1}}
\newcommand{\figref}[1]{Figure~\ref{fig:#1}}
\newcommand{\figrefs}[2]{Figures~\ref{fig:#1} and~\ref{fig:#2}}
\newcommand{\applabel}[1]{\label{app:#1}}
\newcommand{\appref}[1]{Appendix~\ref{app:#1}}
\newcommand{\lnlabel}[1]{\label{line:#1}}
\newcommand{\lnrngref}[2]{lines~\ref{line:#1}--\ref{line:#2}\xspace}
\newcommand{\lnref}[1]{line~\ref{line:#1}\xspace}
\newcommand{\thmref}[1]{Theorem~\ref{thm:#1}\xspace}


% defs FROM ptioa PAPER


\newcommand{\remove}[1]{}
\newcommand{\salg}[1]{\relax\ifmmode {\mathcal F}_{#1}\else ${\mathcal F}_{#1}$\fi} 
\newcommand{\msp}[1]{\relax\ifmmode (#1, \salg{#1}) \else $(#1, \salg{#1})$\fi} 
\newcommand{\msprod}[2]{\relax\ifmmode ( #1 \times #2, \salg{#1} \otimes \salg{#2}) \else $(#1 \times #2, \salg{#1} \otimes \salg{#2})$\fi} 
\newcommand{\dist}[1]{\relax\ifmmode {\mathcal P}\msp{#1}
  \else ${\mathcal P}\msp{#1}$\fi} 
\newcommand{\subdist}[1]{\relax\ifmmode {\mathcal S}{\mathcal P}\msp{#1} 
  \else ${\mathcal S}{\mathcal P}\msp{#1}$\fi} 
\newcommand{\disc}[1]{\relax\ifmmode {\sf Disc}(#1)
  \else ${\sf Disc}(#1)$\fi} 

\newcommand{\Trajeq}{\relax\ifmmode {\mathcal R}_\T \else ${\mathcal R}_\T$\fi} 
\newcommand{\Acteq}{\relax\ifmmode {\mathcal R}_A \else ${\mathcal R}_A$\fi} 
\newcommand{\noop}{\relax\ifmmode \lambda \else $\lambda$\fi} 
\newcommand{\close}[1]{\relax\ifmmode \overline{#1} \else $\overline{#1}$\fi} 

\newcommand{\corrtasks}{\mathop{\mathsf {c}}}
\newcommand{\full}{\mathop{\mathsf {full}}}
%\newcommand{\fstate}{\mathop{\mathsf {fstate}}}
%\newcommand{\lstate}{\mathop{\mathsf {lstate}}}
\newcommand{\tdist}{\mathop{\mathsf {tdist}}}
\newcommand{\extbeh}{\mathop{\mathsf {extbeh}}}
%\newcommand{\apply}{\mathop{\mathsf {apply}}}
\newcommand{\apply}[2]{\mathop{\mathsf {apply}({#1},{#2})}}
%\newcommand{\applytwo}{\mathop{\mathsf {apply2}}}
\newcommand{\support}{\mathop{\mathsf {supp}}}
\newcommand{\relation}{\mathrel{R}}
\newcommand{\cone}{C}
%\newcommand{\tracef}{\mathord{\mathsf {trace}}}
\newcommand{\flatten}{\mathord{\mathsf {flatten}}}
\newcommand{\discrete}{\mathord{\mathsf {Disc}}}
\newcommand{\lift}[1]{\mathrel{{\mathcal L}(#1)}}
\newcommand{\expansion}[1]{\mathrel{{\mathcal E}(#1)}}
%End of commands added by Roberto on May, 12

%>> Ling, December 2005

\newcommand{\subdisc}{\operatorname{\mathsf {SubDisc}}}
\newcommand{\tran}{\operatorname{\mathsf {tran}}}
%\newcommand{\act}{\operatorname{\mathsf {act}}}

\renewcommand{\execs}{{\operatorname{\mathsf {Execs}}}}
\newcommand{\frags}{{\operatorname{\mathsf {Frags}}}}

\newcommand{\tracefnc}{{\operatorname{\mathsf {trace}}}}

\newcommand{\finite}{{\operatorname{\mathsf {finite}}}}
\newcommand{\hide}{{\operatorname{\mathsf {hide}}}}

\newcommand{\early}{{\operatorname{\mathsf {Early}}}}
\newcommand{\late}{{\operatorname{\mathsf {Late}}}}
\newcommand{\toss}{{\operatorname{\mathsf {Toss}}}}

\newcommand{\define}{:=}

\newcommand{\pc}{{\operatorname{\mathsf {counter}}}}
\newcommand{\chosen}{{\operatorname{\mathsf {chosen}}}}

\newcommand{\rand}{{\operatorname{\mathsf {random}}}}
\newcommand{\unif}{{\operatorname{\mathsf {unif}}}}

\newcommand{\ie}{i.e.,\xspace}
\newcommand{\Ie}{I.e.,\xspace}

\newcommand{\eg}{e.g.,\xspace}
\newcommand{\Eg}{E.g.,\xspace}

% IOA related stuff




\newcommand{\mybox}[3]{
  \framebox[#1][l]
  {
    \parbox{#2}
    {
      #3
    }
  }
}

\newcommand{\two}[4]{
  \parbox{.95\columnwidth}{\vspace{1pt} \vfill
    \parbox[t]{#1\columnwidth}{#3}%
    \parbox[t]{#2\columnwidth}{#4}%
  }}

\newcommand{\twosep}[4]{
  \parbox{\columnwidth}{\vspace{1pt} \vfill
    \parbox[t]{#1\columnwidth}{#3}%
   	\vrule width 0.2pt
    \parbox[t]{#2\columnwidth}{#4}%
  }}

\newcommand{\eqntwo}[4]{
  \parbox{\columnwidth}{\vspace{1pt} \vfill
    \parbox[t]{#1\columnwidth}{$ #3 $}
    \parbox[t]{#2\columnwidth}{$ #4 $}
  }}

\newcommand{\three}[6]{\vspace{1pt} \vfill
        \parbox{\columnwidth}{%
        \parbox[t]{#1\columnwidth}{#4}%
        \parbox[t]{#2\columnwidth}{#5}%
        \parbox[t]{#3\columnwidth}{#6}%
      }}

\newcommand{\tup}[1]
           {
             \relax\ifmmode
             \langle #1 \rangle
             \else $\langle$ #1 $\rangle$ \fi
           }

\newcommand{\lit}[1]{ \relax\ifmmode
                \mathord{\mathcode`\-="702D\sf #1\mathcode`\-="2200}
                \else {\it #1} \fi }


\newcommand{\figuresize}{\scriptsize}

\newcommand{\equationsize}{\footnotesize}

%\newcommand{\act}[1]{%
%  \relax\ifmmode
%                \mathord{\mathcode`\-="702D\sf #1\mathcode`\-="2200}%
%                \else
%                {\sf #1\/}%
%                \fi }


\lstdefinelanguage{ioa}{
  basicstyle=\figuresize,
  keywordstyle=\bf \figuresize,
  identifierstyle=\it \figuresize,
  emphstyle=\tt \figuresize,
  mathescape=true,
  tabsize=20,
%  tabsize=4,
  sensitive=false,
  columns=fullflexible,
  keepspaces=false,
  flexiblecolumns=true,
%  basewidth=0.5em,
  basewidth=0.05em,
  moredelim=[il][\rm]{//},
  moredelim=[is][\sf \figuresize]{!}{!},
  moredelim=[is][\bf \figuresize]{*}{*},
  keywords={automaton,and, 
  	 choose,const,continue, components,
  	 discrete, do,
  	 eff, external,else, elseif, evolve, end,
  	 fi,for, forward, from,
  	 hidden,
  	 in,input,internal,if,invariant, initially, imports,
     let,
     or, output, operators, od, of,
     pre,
     return,
     such,satisfies, stop, signature, simulation, 
     trajectories,trajdef, transitions, that,then, type, types, to, tasks,
     variables, vocabulary, 
     when,where, with,while},
  emph={set, seq, tuple, map, array, enumeration},   
   literate=
        {(}{{$($}}1
        {)}{{$)$}}1
        % LaTeX math symbols
        {\\in}{{$\in\ $}}1
        {\\preceq}{{$\preceq\ $}}1
        {\\subset}{{$\subset\ $}}1
        {\\subseteq}{{$\subseteq\ $}}1
        {\\supset}{{$\supset\ $}}1
        {\\supseteq}{{$\supseteq\ $}}1
        {\\forall}{{$\forall$}}1
        {\\le}{{$\le\ $}}1
        {\\ge}{{$\ge\ $}}1
        {\\gets}{{$\gets\ $}}1
        {\\cup}{{$\cup\ $}}1
        {\\cap}{{$\cap\ $}}1
        {\\langle}{{$\langle$}}1
        {\\rangle}{{$\rangle$}}1
        {\\exists}{{$\exists\ $}}1
        {\\bot}{{$\bot$}}1
        {\\rip}{{$\rip$}}1
        {\\emptyset}{{$\emptyset$}}1
        {\\notin}{{$\notin\ $}}1
        {\\not\\exists}{{$\not\exists\ $}}1
        {\\ne}{{$\ne\ $}}1
        {\\to}{{$\to\ $}}1
        {\\implies}{{$\implies\ $}}1
        % LSL symbols (one-character)
        {<}{{$<\ $}}1
        {>}{{$>\ $}}1
        {=}{{$=\ $}}1
        {~}{{$\neg\ $}}1
        {|}{{$\mid$}}1
        {'}{{$^\prime$}}1
        % LSL symbols (two characters)
        {\\A}{{$\forall\ $}}1
        {\\E}{{$\exists\ $}}1
        {\\/}{{$\vee\,$}}1
        {\\vee}{{$\vee\,$}}1
        {/\\}{{$\wedge\,$}}1
        {\\wedge}{{$\wedge\,$}}1
        {=>}{{$\Rightarrow\ $}}1
        {->}{{$\rightarrow\ $}}1
        {<=}{{$\Leftarrow\ $}}1
        {<-}{{$\leftarrow\ $}}1
%        {<=}{{$\leq$}}1
%        {>=}{{$\geq$}}1
        {~=}{{$\neq\ $}}1
        {\\U}{{$\cup\ $}}1
        {\\I}{{$\cap\ $}}1
        {|-}{{$\vdash\ $}}1
        {-|}{{$\dashv\ $}}1
        {<<}{{$\ll\ $}}2
        {>>}{{$\gg\ $}}2
        {||}{{$\|$}}1
%%       {\[\]}{{\[\,\]}}2 {\{\}}{{\{\,\}}}2
%%        {[}{{$\langle$}}1
%%        {]}{{$\rangle$}}1
        {[}{{$[$}}1
        {]}{{$\,]$}}1
        {[[}{{$\langle$}}1
        {]]]}{{$]\rangle$}}1
        {]]}{{$\rangle$}}1
        {<=>}{{$\Leftrightarrow\ $}}2
        {<->}{{$\leftrightarrow\ $}}2
        {(+)}{{$\oplus\ $}}1
        {(-)}{{$\ominus\ $}}1
        {_i}{{$_{i}$}}1
        {_j}{{$_{j}$}}1
        {_{i,j}}{{$_{i,j}$}}3
        {_{j,i}}{{$_{j,i}$}}3
        {_0}{{$_0$}}1
        {_1}{{$_1$}}1
        {_2}{{$_2$}}1
        {_n}{{$_n$}}1
        {_p}{{$_p$}}1
        {_k}{{$_n$}}1
        {-}{{$\ms{-}$}}1
        {@}{{}}0
        {\\delta}{{$\delta$}}1
        {\\R}{{$\R$}}1
        {\\Rplus}{{$\Rplus$}}1
        {\\N}{{$\N$}}1
        {\\times}{{$\times\ $}}1
        {\\tau}{{$\tau$}}1
        {\\alpha}{{$\alpha$}}1
        {\\beta}{{$\beta$}}1
        {\\gamma}{{$\gamma$}}1
        {\\ell}{{$\ell\ $}}1
        {--}{{$-\ $}}1
        {\\TT}{{\hspace{1.5em}}}3        
      }

\lstdefinelanguage{ioaNums}[]{ioa}
{
  numbers=left,
  numberstyle=\tiny,
  stepnumber=2,
  numbersep=4pt
%  firstnumber=1
}

\lstdefinelanguage{ioaNumsRight}[]{ioa}
{
  numbers=right,
  numberstyle=\tiny,
  stepnumber=2,
  numbersep=4pt
%  firstnumber=1
}

\newcommand{\ioa}{\lstinline[language=IOA]}

\lstnewenvironment{IOA}%
  {\lstset{language=IOA}}
  {}

\lstnewenvironment{IOANums}%
  {
  \if@firstcolumn
    \lstset{language=IOA, numbers=left, firstnumber=auto}
  \else
    \lstset{language=IOA, numbers=right, firstnumber=auto}
  \fi
  }
  {}

\lstnewenvironment{IOANumsRight}%
  {
    \lstset{language=IOA, numbers=right, firstnumber=auto}
  }
  {}

%\lstnewenvironment{IOA}%
%  {\lstset{language=ioaLang}
%   \csname lst@SetFirstLabel\endcsname}
%  {\csname lst@SaveFirstLabel\endcsname\vspace{-4pt}\noindent}

\newcommand{\figioa}[5]{
  \begin{figure}[#1]
      \hrule \F
      {\figuresize \bf #2}
      \lstinputlisting[language=ioaLang]{#5}
      \F \hrule \F
      \caption{#3}
      \label{fig: #4}
  \end{figure}
}

\newcommand{\linefigioa}[9]{

}

\newcommand{\twofigioa}[8]{
  \begin{figure}[#1]
    \hrule \F
    {\figuresize \bf #2} \\
    \two{#5}{#6}
    {
      \lstinputlisting[language=ioaLang]{#7}
    }
    {
      \lstinputlisting[language=ioaLang]{#8}
    }
    \F \hrule \F
    \caption{#3}
    \label{fig: #4}
  \end{figure}
}



\lstdefinelanguage{ioaLang}{%
  basicstyle=\ttfamily\small,
  keywordstyle=\rmfamily\bfseries\small,
  identifierstyle=\small,
%  commentline=\%,
  keywords={assumes,automaton,axioms,backward,bounds,by,case,choose,components,const,d,det,discrete,do,eff,else,elseif,ensuring,enumeration,evolve,fi,fire,follow,for,forward,from,hidden,if,in,%
    input,initially,internal,invariant,let, local,od,of,output,pre,schedule,signature,so,%
    simulation,states,variables, tasks, stop,tasks,that,then,to,trajdef,trajectory,trajectories,transitions,tuple,type,union,urgent,uses,when,where,while,yield},
  literate=
        % LaTeX math symbols
        {\\in}{{$\in$}}1
        {\\preceq}{{$\preceq$}}1
        {\\subset}{{$\subset$}}1
        {\\subseteq}{{$\subseteq$}}1
        {\\supset}{{$\supset$}}1
        {\\supseteq}{{$\supseteq$}}1
        {\\rho}{{$\rho$}}1
        {\\infty}{{$\infty$}}1
        % LSL symbols (one-character)
        {<}{{$<$}}1
        {>}{{$>$}}1
        {=}{{$=$}}1
        {~}{{$\neg$}}1 
        {|}{{$\mid$}}1
        {'}{{$^\prime$}}1
        % LSL symbols (two characters)
        {\\A}{{$\forall$}}1 {\\E}{{$\exists$}}1
        {\\/}{{$\vee$}}1 {/\\}{{$\wedge$}}1 
        {=>}{{$\Rightarrow$}}1 
        {->}{{$\rightarrow$}}1 
        {<=}{{$\leq$}}1 {>=}{{$\geq$}}1 {~=}{{$\neq$}}1
        {\\U}{{$\cup$}}1 {\\I}{{$\cap$}}1
        {|-}{{$\vdash$}}1 {-|}{{$\dashv$}}1
        {<<}{{$\ll$}}2 {>>}{{$\gg$}}2
        {||}{{$\|$}}1
%       {\[\]}{{\[\,\]}}2 {\{\}}{{\{\,\}}}2
        % LSL symbols (three or more characters)
        {<=>}{{$\Leftrightarrow$}}2 
        {<->}{{$\leftrightarrow$}}2
        {(+)}{{$\oplus$}}1
        {(-)}{{$\ominus$}}1
}

\lstdefinelanguage{bigIOALang}{%
  basicstyle=\ttfamily,
  keywordstyle=\rmfamily\bfseries,
  identifierstyle=,
%  commentline=\%,
  keywords={assumes,automaton,axioms,backward,by,case,choose,components,const,%
    d,det,discrete,do,eff,else,elseif,ensuring,enumeration,evolve,fi,for,forward,from,hidden,if,in%
    input,initially,internal,invariant,local,od,of,output,pre,schedule,signature,so,%
    tasks, simulation,states,stop,tasks,that,then,to,trajdef,trajectories,transitions,tuple,type,union,urgent,uses,when,where,yield},
  literate=
        % LaTeX math symbols
        {\\in}{{$\in$}}1
        {\\preceq}{{$\preceq$}}1
        {\\subset}{{$\subset$}}1
        {\\subseteq}{{$\subseteq$}}1
        {\\supset}{{$\supset$}}1
        {\\supseteq}{{$\supseteq$}}1
        % LSL symbols (one-character)
        {<}{{$<$}}1
        {>}{{$>$}}1
        {=}{{$=$}}1
        {~}{{$\neg$}}1 
        {|}{{$\mid$}}1
        {'}{{$^\prime$}}1
        % LSL symbols (two characters)
        {\\A}{{$\forall$}}1 {\\E}{{$\exists$}}1
        {\\/}{{$\vee$}}1 {/\\}{{$\wedge$}}1 
        {=>}{{$\Rightarrow$}}1 
        {->}{{$\rightarrow$}}1 
        {<=}{{$\leq$}}1 {>=}{{$\geq$}}1 {~=}{{$\neq$}}1
        {\\U}{{$\cup$}}1 {\\I}{{$\cap$}}1
        {|-}{{$\vdash$}}1 {-|}{{$\dashv$}}1
        {<<}{{$\ll$}}2 {>>}{{$\gg$}}2
        {||}{{$\|$}}1
%       {\[\]}{{\[\,\]}}2 {\{\}}{{\{\,\}}}2
        % LSL symbols (three or more characters)
        {<=>}{{$\Leftrightarrow$}}2 
        {<->}{{$\leftrightarrow$}}2
        {(+)}{{$\oplus$}}1
        {(-)}{{$\ominus$}}1
}


\lstnewenvironment{BigIOA}%
  {\lstset{language=bigIOALang,basicstyle=\ttfamily}
   \csname lst@SetFirstLabel\endcsname}
  {\csname lst@SaveFirstLabel\endcsname\vspace{-4pt}\noindent}

\lstnewenvironment{SmallIOA}%
  {\lstset{language=ioaLang,basicstyle=\ttfamily\scriptsize}
   \csname lst@SetFirstLabel\endcsname}
  %{\csname lst@SaveFirstLabel\endcsname\vspace{-4pt}\noindent}
  {\csname lst@SaveFirstLabel\endcsname\noindent}


\newcommand{\true}{\relax\ifmmode \mathit true \else \em true \/\fi}
\newcommand{\false}{\relax\ifmmode \mathit false \else \em false \/\fi}



\newcommand{\Real}{{\operatorname{\texttt{Real}}}}
\newcommand{\Bool}{{\operatorname{\texttt{Bool}}}}
\newcommand{\Char}{{\operatorname{\texttt{Char}}}}
\newcommand{\ioaInt}{{\operatorname{\texttt{Int}}}}
\newcommand{\ioaNat}{{\operatorname{\texttt{Nat}}}}
\newcommand{\ioaAugR}{{\operatorname{\texttt{AugmentedReal}}}}
\newcommand{\ioaString}{{\operatorname{\texttt{String}}}}
\newcommand{\Discrete}{{\operatorname{\texttt{Discrete}}}}
%\newcommand{\Bool}{{\operatorname{\texttt{Bool}}}}

%\newcommand{\lnot}{\neg}
%\newcommand{\land}{\wedge}
%\newcommand{\lor}{\vee}
\newcommand{\limplies}{\Rightarrow}
\newcommand{\liff}{\Leftrightarrow}

\newlength{\bracklen}
\newcommand{\sem}[1]{\settowidth{\bracklen}{[}
     [\hspace{-0.5\bracklen}[#1]\hspace{-0.5\bracklen}]}

\newcommand{\defaultArraystretch}{1.4}
\renewcommand{\arraystretch}{\defaultArraystretch}

\newcommand{\gS}{\mathcal{S}}
\newcommand{\gV}{\mathcal{V}}
\newcommand{\freevars}{\mathcal{FV}}

\newcommand{\gVspec}{\mathcal{V}_\mathit{spec}}
\newcommand{\gVa}{\mathcal{V}_\mathit{A}}
\newcommand{\gVsig}{\mathcal{V}_\mathit{sigs}}
\newcommand{\gVso}{\mathcal{V}_\mathit{sorts}}
\newcommand{\gVop}{\mathcal{V}_\mathit{ops}}
\newcommand{\sort}{\mathit{sort}}
\newcommand{\sig}{\mathit{sig}}
\newcommand{\id}{\mathit{id}}
\newcommand{\sigsep}{\lsl`->`}

%\newcommand{\T}{\mathit{true}}
%\newcommand{\F}{\mathit{false}}

\newcommand{\super}[2]{\ensuremath{\mathit{#1}^\mathit{#2}}}
\newcommand{\tri}[3]{\ensuremath{\mathit{#1}^\mathit{#2}_\mathit{#3}}}
\newcommand{\Assumptions}{\ensuremath{\mathit{Assumptions}}}
\newcommand{\actPred}[3][\pi]{\tri{P}{#2,#1}{#3}}
\newcommand{\actualTypes}[1]{\super{actualTypes}{#1}}
\newcommand{\actuals}[1]{\super{actuals}{#1}}
\newcommand{\autActVars}[2][\pi]{\vars{#2}{},\vars{#2,#1}{}}
\newcommand{\bracket}[2]{\mathit{#1}[\mathit{#2}]}
\newcommand{\compVars}[1]{\super{compVars}{#1}}
\newcommand{\context}{\mathit{context}}
\newcommand{\ensuring}[2]{\tri{ensuring}{#1}{#2}}
\newcommand{\initPred}[1]{\tri{P}{#1}{init}}
\newcommand{\initVals}[1]{\super{initVals}{#1}}
\newcommand{\initially}[2]{\tri{initially}{#1}{#2}}
\newcommand{\invPred}[2]{\tri{Inv}{#1}{#2}}
\newcommand{\knownVars}[1]{\super{knownVars}{#1}}
\newcommand{\localPostVars}[2]{\tri{localPostVars}{#1}{#2}}
\newcommand{\localVars}[2]{\tri{localVars}{#1}{#2}}
\newcommand{\locals}[1]{\bracket{Locals}{#1}}
\newcommand{\nam}[1]{\rho^{\mathit{#1}}}
\newcommand{\otherActPred}[3][\pi]{\otherTri{P}{#2,#1}{#3}}
\newcommand{\otherParams}[2]{\otherTri{params}{#1}{#2}}
\newcommand{\otherSub}[2]{\otherTri{\sigma}{#1}{#2}}
\newcommand{\otherTri}[3]{\tri{\smash{#1'}}{#2}{#3}}
\newcommand{\otherVars}[2]{\otherTri{vars}{#1}{#2}}
\newcommand{\params}[2]{\tri{params}{#1}{#2}}
\newcommand{\postVars}[1]{\super{postVars}{#1}}
\newcommand{\pre}[2]{\tri{Pre}{#1}{#2}}
\newcommand{\prog}[2]{\tri{Prog}{#1}{#2}}
\newcommand{\prov}[2]{\tri{Prov}{#1}{#2}}
\newcommand{\stateSorts}[1]{\super{stateSorts}{#1}}
\newcommand{\stateVars}[1]{\super{stateVars}{#1}}
\newcommand{\states}[1]{\bracket{States}{#1}}
\newcommand{\sub}[2]{\tri{\sigma}{#1}{#2}}
\newcommand{\sugActPred}[3][\pi]{\tri{P}{#2,#1}{#3,desug}}
\newcommand{\sugLocalVars}[2]{\ifthenelse{\equal{}{#2}}%
                             {\tri{localVars}{#1}{desug}}%
                             {\tri{localVars}{#1}{#2,desug}}}
\newcommand{\sugVars}[2]{\ifthenelse{\equal{}{#2}}%
                        {\tri{vars}{#1}{desug}}%
                        {\tri{vars}{#1}{#2,desug}}}
\newcommand{\cVars}[1]{\super{cVars}{#1}}

\newcommand{\vmap}{\dot{\varrho}}
\newcommand{\map}[2]{\tri{\vmap}{#1}{#2}}

\newcommand{\types}[1]{\super{types}{#1}}
\newcommand{\vars}[2]{\tri{vars}{#1}{#2}}

\newcommand{\subActPred}[3][\pi]{\sub{#2,#1}{#3}(\tri{P}{#2,#1}{#3,desug})}
\newcommand{\subLocalVars}[2]{\sub{#1}{#2}(\tri{localVars}{#1}{#2,desug})}

\newcommand{\dA}{\hat{A}}
\newcommand{\renameAction}[1]{\ensuremath{\rho_{#1}(\vars{\dA{#1},\pi}{})}}
\newcommand{\renameComponent}[1]{\ensuremath{\rho_{#1}\dA_{#1}}}

\newenvironment{Syntax}{\[\begin{subSyntax}}{\end{subSyntax}\]\vspace{-.3in}}
\newenvironment{subSyntax}{\begin{array}{l}}{\end{array}}
\newcommand{\w}[1]{\mbox{\hspace*{#1em}}}



%TIOA macros
% Script math symbol name
\newcommand{\ms}[1]{\ifmmode%
\mathord{\mathcode`-="702D\it #1\mathcode`\-="2200}\else%
$\mathord{\mathcode`-="702D\it #1\mathcode`\-="2200}$\fi}

%\newcommand{\domain}[1]{{\it dom}(#1)} % domain

% KEYWORDS 
\newcommand{\kw}[1]{{\bf #1}} 
\newcommand{\tcon}[1]{{\tt #1}} 
\newcommand{\syn}[1]{{\tt #1}} 
\newcommand{\pvskw}[1]{{\sc #1}} 
\newcommand{\pvsid}[1]{{\operatorname{\mathit{#1}}}}


% TIMED AUTOMATA
\def\A{{\cal A}} % TA
\def\B{{\cal B}} % TA
\def\D{{\cal D}} % set of discrete steps
\def\T{{\cal T}} % set of trajectories

% VALUATIONS
\newcommand{\vv}{{\bf v}}
\newcommand{\vw}{{\bf w}}
\newcommand{\vx}{{\bf x}}
\newcommand{\vy}{{\bf y}}
\newcommand{\va}{{\bf a}}
\newcommand{\vb}{{\bf b}}
\newcommand{\vq}{{\bf q}}
\newcommand{\vs}{{\bf s}}
\newcommand{\vm}{{\bf m}}

% Transitions and trajectory operations
\newcommand{\arrow}[1]{\mathrel{\stackrel{#1}{\rightarrow}}}
\newcommand{\sarrow}[2]{\mathrel{\stackrel{#1}{\rightarrow_{#2}}}}
\newcommand{\concat}{\mathbin{^{\frown}}} % concatenation
\newcommand{\paste}{\mathrel{\diamond}}

% AUTOMATA
%\def\A{{\cal A}}
%\def\B{{\cal B}}
%\def\T{{\cal T}} % set of trajectories

\def\CC{{\mathscr C}} % concatenation closure

% PVS STUFF

\lstdefinelanguage{pvs}{
  basicstyle=\tt \figuresize,
  keywordstyle=\sc \figuresize,
  identifierstyle=\it \figuresize,
  emphstyle=\tt \figuresize,
  mathescape=true,
  tabsize=20,
%  tabsize=4,
  sensitive=false,
  columns=fullflexible,
  keepspaces=false,
  flexiblecolumns=true,
%  basewidth=0.5em,
  basewidth=0.05em,
  moredelim=[il][\rm]{//},
  moredelim=[is][\sf \figuresize]{!}{!},
  moredelim=[is][\bf \figuresize]{*}{*},
  keywords={and, 
  	 begin,
  	 cases, const,
  	 do,
  	 external, else, exists, end, endcases, endif,
  	 fi,for, forall, from,
  	 hidden,
  	 in, if, importing,
     let, lambda, lemma,
     measure, 
     not,
     or, of,
     return, recursive,
     stop, 
     theory, that,then, type, types, type+, to, theorem,
     var,
     with,while},
  emph={nat, setof, sequence, eq, tuple, map, array, enumeration, bool, real, exp, nnreal, posreal},   
   literate=
        {(}{{$($}}1
        {)}{{$)$}}1
        % LaTeX math symbols
        {\\in}{{$\in\ $}}1
        {\\mapsto}{{$\rightarrow\ $}}1
        {\\preceq}{{$\preceq\ $}}1
        {\\subset}{{$\subset\ $}}1
        {\\subseteq}{{$\subseteq\ $}}1
        {\\supset}{{$\supset\ $}}1
        {\\supseteq}{{$\supseteq\ $}}1
        {\\forall}{{$\forall$}}1
        {\\le}{{$\le\ $}}1
        {\\ge}{{$\ge\ $}}1
        {\\gets}{{$\gets\ $}}1
        {\\cup}{{$\cup\ $}}1
        {\\cap}{{$\cap\ $}}1
        {\\langle}{{$\langle$}}1
        {\\rangle}{{$\rangle$}}1
        {\\exists}{{$\exists\ $}}1
        {\\bot}{{$\bot$}}1
        {\\rip}{{$\rip$}}1
        {\\emptyset}{{$\emptyset$}}1
        {\\notin}{{$\notin\ $}}1
        {\\not\\exists}{{$\not\exists\ $}}1
        {\\ne}{{$\ne\ $}}1
        {\\to}{{$\to\ $}}1
        {\\implies}{{$\implies\ $}}1
        % LSL symbols (one-character)
        {<}{{$<\ $}}1
        {>}{{$>\ $}}1
        {=}{{$=\ $}}1
        {~}{{$\neg\ $}}1
        {|}{{$\mid$}}1
        {'}{{$^\prime$}}1
        % LSL symbols (two characters)
        {\\A}{{$\forall\ $}}1
        {\\E}{{$\exists\ $}}1
        {\\/}{{$\vee\,$}}1
        {\\vee}{{$\vee\,$}}1
        {/\\}{{$\wedge\,$}}1
        {\\wedge}{{$\wedge\,$}}1
        {->}{{$\rightarrow\ $}}1
        {=>}{{$\Rightarrow\ $}}1
        {->}{{$\rightarrow\ $}}1
        {<=}{{$\Leftarrow\ $}}1
        {<-}{{$\leftarrow\ $}}1
%        {<=}{{$\leq$}}1
%        {>=}{{$\geq$}}1
        {~=}{{$\neq\ $}}1
        {\\U}{{$\cup\ $}}1
        {\\I}{{$\cap\ $}}1
        {|-}{{$\vdash\ $}}1
        {-|}{{$\dashv\ $}}1
        {<<}{{$\ll\ $}}2
        {>>}{{$\gg\ $}}2
        {||}{{$\|$}}1
%%       {\[\]}{{\[\,\]}}2 {\{\}}{{\{\,\}}}2
%%        {[}{{$\langle$}}1
%%        {]}{{$\rangle$}}1
        {[}{{$[$}}1
        {]}{{$\,]$}}1
        {[[}{{$\langle$}}1
        {]]]}{{$]\rangle$}}1
        {]]}{{$\rangle$}}1
        {<=>}{{$\Leftrightarrow\ $}}2
        {<->}{{$\leftrightarrow\ $}}2
        {(+)}{{$\oplus\ $}}1
        {(-)}{{$\ominus\ $}}1
        {_i}{{$_{i}$}}1
        {_j}{{$_{j}$}}1
        {_{i,j}}{{$_{i,j}$}}3
        {_{j,i}}{{$_{j,i}$}}3
        {_0}{{$_0$}}1
        {_1}{{$_1$}}1
        {_2}{{$_2$}}1
        {_n}{{$_n$}}1
        {_p}{{$_p$}}1
        {_k}{{$_n$}}1
        {-}{{$\ms{-}$}}1
        {@}{{}}0
        {\\delta}{{$\delta$}}1
        {\\R}{{$\R$}}1
        {\\Rplus}{{$\Rplus$}}1
        {\\N}{{$\N$}}1
        {\\times}{{$\times\ $}}1
        {\\tau}{{$\tau$}}1
        {\\alpha}{{$\alpha$}}1
        {\\beta}{{$\beta$}}1
        {\\gamma}{{$\gamma$}}1
        {\\ell}{{$\ell\ $}}1
        {--}{{$-\ $}}1
        {\\TT}{{\hspace{1.5em}}}3        
      }



\lstdefinelanguage{BigPVS}{
  basicstyle=\tt,
  keywordstyle=\sc,
  identifierstyle=\it,
  emphstyle=\tt ,
  mathescape=true,
  tabsize=20,
%  tabsize=4,
  sensitive=false,
  columns=fullflexible,
  keepspaces=false,
  flexiblecolumns=true,
%  basewidth=0.5em,
  basewidth=0.05em,
  moredelim=[il][\rm]{//},
  moredelim=[is][\sf \figuresize]{!}{!},
  moredelim=[is][\bf \figuresize]{*}{*},
  keywords={and, 
  	 begin,
  	 cases, const,
  	 do, datatype,
  	 external, else, exists, end, endif, endcases,
  	 fi,for, forall, from,
  	 hidden,
  	 in, if, importing,
     let, lambda, lemma,
     measure,
     not,
     or, of,
     return, recursive,
     stop, 
     theory, that,then, type, types, type+, to, theorem,
     var,
     with,while},
  emph={nat, setof, sequence, eq, tuple, map, array, first, rest, add, enumeration, bool, real, posreal, nnreal},   
   literate=
        {(}{{$($}}1
        {)}{{$)$}}1
        % LaTeX math symbols
        {\\in}{{$\in\ $}}1
        {\\mapsto}{{$\rightarrow\ $}}1
        {\\preceq}{{$\preceq\ $}}1
        {\\subset}{{$\subset\ $}}1
        {\\subseteq}{{$\subseteq\ $}}1
        {\\supset}{{$\supset\ $}}1
        {\\supseteq}{{$\supseteq\ $}}1
        {\\forall}{{$\forall$}}1
        {\\le}{{$\le\ $}}1
        {\\ge}{{$\ge\ $}}1
        {\\gets}{{$\gets\ $}}1
        {\\cup}{{$\cup\ $}}1
        {\\cap}{{$\cap\ $}}1
        {\\langle}{{$\langle$}}1
        {\\rangle}{{$\rangle$}}1
        {\\exists}{{$\exists\ $}}1
        {\\bot}{{$\bot$}}1
        {\\rip}{{$\rip$}}1
        {\\emptyset}{{$\emptyset$}}1
        {\\notin}{{$\notin\ $}}1
        {\\not\\exists}{{$\not\exists\ $}}1
        {\\ne}{{$\ne\ $}}1
        {\\to}{{$\to\ $}}1
        {\\implies}{{$\implies\ $}}1
        % LSL symbols (one-character)
        {<}{{$<\ $}}1
        {>}{{$>\ $}}1
        {=}{{$=\ $}}1
        {~}{{$\neg\ $}}1
        {|}{{$\mid$}}1
        {'}{{$^\prime$}}1
        % LSL symbols (two characters)
        {\\A}{{$\forall\ $}}1
        {\\E}{{$\exists\ $}}1
        {\\/}{{$\vee\,$}}1
        {\\vee}{{$\vee\,$}}1
        {/\\}{{$\wedge\,$}}1
        {\\wedge}{{$\wedge\,$}}1
        {->}{{$\rightarrow\ $}}1
        {=>}{{$\Rightarrow\ $}}1
        {->}{{$\rightarrow\ $}}1
        {<=}{{$\Leftarrow\ $}}1
        {<-}{{$\leftarrow\ $}}1
%        {<=}{{$\leq$}}1
%        {>=}{{$\geq$}}1
        {~=}{{$\neq\ $}}1
        {\\U}{{$\cup\ $}}1
        {\\I}{{$\cap\ $}}1
        {|-}{{$\vdash\ $}}1
        {-|}{{$\dashv\ $}}1
        {<<}{{$\ll\ $}}2
        {>>}{{$\gg\ $}}2
        {||}{{$\|$}}1
%%       {\[\]}{{\[\,\]}}2 {\{\}}{{\{\,\}}}2
%%        {[}{{$\langle$}}1
%%        {]}{{$\rangle$}}1
        {[}{{$[$}}1
        {]}{{$\,]$}}1
        {[[}{{$\langle$}}1
        {]]]}{{$]\rangle$}}1
        {]]}{{$\rangle$}}1
        {<=>}{{$\Leftrightarrow\ $}}2
        {<->}{{$\leftrightarrow\ $}}2
        {(+)}{{$\oplus\ $}}1
        {(-)}{{$\ominus\ $}}1
        {_i}{{$_{i}$}}1
        {_j}{{$_{j}$}}1
        {_{i,j}}{{$_{i,j}$}}3
        {_{j,i}}{{$_{j,i}$}}3
        {_0}{{$_0$}}1
        {_1}{{$_1$}}1
        {_2}{{$_2$}}1
        {_n}{{$_n$}}1
        {_p}{{$_p$}}1
        {_k}{{$_n$}}1
        {-}{{$\ms{-}$}}1
        {@}{{}}0
        {\\delta}{{$\delta$}}1
        {\\R}{{$\R$}}1
        {\\Rplus}{{$\Rplus$}}1
        {\\N}{{$\N$}}1
        {\\times}{{$\times\ $}}1
        {\\tau}{{$\tau$}}1
        {\\alpha}{{$\alpha$}}1
        {\\beta}{{$\beta$}}1
        {\\gamma}{{$\gamma$}}1
        {\\ell}{{$\ell\ $}}1
        {--}{{$-\ $}}1
        {\\TT}{{\hspace{1.5em}}}3        
      }

\lstdefinelanguage{pvsNums}[]{pvs}
{
  numbers=left,
  numberstyle=\tiny,
  stepnumber=2,
  numbersep=4pt
%  firstnumber=1
}

\lstdefinelanguage{pvsNumsRight}[]{pvs}
{
  numbers=right,
  numberstyle=\tiny,
  stepnumber=2,
  numbersep=4pt
%  firstnumber=1
}

\newcommand{\pvs}{\lstinline[language=PVS]}

\lstnewenvironment{BigPVS}%
  {\lstset{language=BigPVS}}
  {}

\lstnewenvironment{PVSNums}%
  {
  \if@firstcolumn
    \lstset{language=pvs, numbers=left, firstnumber=auto}
  \else
    \lstset{language=pvs, numbers=right, firstnumber=auto}
  \fi
  }
  {}

\lstnewenvironment{PVSNumsRight}%
  {
    \lstset{language=pvs, numbers=right, firstnumber=auto}
  }
  {}


\newcommand{\figpvs}[5]{
  \begin{figure}[#1]
      \hrule \F
      {\figuresize \bf #2}
      \lstinputlisting[language=pvs]{#5}
      \F \hrule \F
      \caption{#3}
      \label{fig: #4}
  \end{figure}
}

\newcommand{\linefigpvs}[9]{

}

\newcommand{\twofigpvs}[8]{
  \begin{figure}[#1]
    \hrule \F
    {\figuresize \bf #2} \\
    \two{#5}{#6}
    {
      \lstinputlisting[language=pvsLang]{#7}
    }
    {
      \lstinputlisting[language=pvsLang]{#8}
    }
    \F \hrule \F
    \caption{#3}
    \label{fig: #4}
  \end{figure}
}


\lstdefinelanguage{pvsproof}{
  basicstyle=\tt \figuresize,
  mathescape=true,
  tabsize=4,
  sensitive=false,
  columns=fullflexible,
  keepspaces=false,
  flexiblecolumns=true,
  basewidth=0.05em,
}

\newcommand{\Giant}{\fontsize{120}{130}\selectfont}

\newcommand{\CrazyBig}{\fontsize{200}{210}\selectfont}



\begin{frame}
\titlepage
\end{frame}


\begin{frame}{Inverted Pendulum}


\begin{center}
\begin{figure}[h!]
	\scalebox{1} {
		\begin{pspicture}(0,-2.94)(6.12,2.94)
		\psframe[linewidth=0.04,dimen=outer](5.42,0.56)(1.08,-1.1)
		\pscircle[linewidth=0.04,dimen=outer](2.09,-1.57){0.47}
		\psline[linewidth=0.04cm](0.0,-2.02)(6.1,-2.02)
		\pscircle[linewidth=0.04,dimen=outer](4.14,-1.56){0.48}
		\psline[linewidth=0.04cm,linestyle=dotted,dotsep=0.16cm](3.02,2.86)(3.0,0.56)
		\psline[linewidth=0.08cm](3.02,0.52)(4.2,2.9)
		\usefont{T1}{ptm}{m}{n}
		\rput(3.4390626,2.3){\large $\theta$}
		\usefont{T1}{ptm}{m}{n}
		\rput(3.298125,-0.295){\large M}
		\usefont{T1}{ptm}{m}{n}
		\rput(4.4246874,1.72){\large m, l}
		\psline[linewidth=0.04cm](0.2,-2.06)(0.2,-2.84)
		\psline[linewidth=0.04cm,arrowsize=0.05291667cm 2.0,arrowlength=1.4,arrowinset=0.4]{->}(0.24,-2.4)(4.32,-2.4)
		\usefont{T1}{ptm}{m}{n}
		\rput(1.8825,-2.72){\large x}
	\end{pspicture}}
	\caption{Inverted Pendulum System}
	\label{fig:pendulumSystem}
\end{figure}
\end{center}


\end{frame}

\begin{frame}{Nonlinear System Model}
\begin{itemize}
\pause \item $\dot{X}=f\left(X,u\right)$
\pause \item Equations of Motion for Plant\\
$\left(m+M\right)\ddot{x}+\frac{1}{2}ml\cos\left(\theta\right)\ddot{\theta}-\frac{1}{2}ml\sin\left(\theta\right)\dot{\theta}^2=F-f_c$\\
$\frac{1}{2}ml\mathsf{\theta}\ddot{x}+\frac{1}{3}ml^2\ddot{\theta}-\frac{1}{2}mgl\sin\left(\theta\right)=-f_p$
\pause \item Full System Model (including Motor Dynamics)\\
$\ddot{x}=\frac{1}{D}\left[\frac{1}{3}ml^2\left(B_lV_a-f_c-C_1\right)+\frac{1}{2}ml\cos\left(\theta\right)\left(f_p+C_2\right)\right]$
$\ddot{\theta}=\frac{1}{D}\left[-\frac{1}{2}ml\cos\left(\theta\right)\left(B_lV_a-f_c-C_1\right)-\bar{M}\left(f_p+C_2\right)\right]$ \\
\pause where
$D=\frac{1}{3}\bar{M}ml^2-\frac{1}{4}m^2l^2\cos^2\left(\theta\right)$\\
$\bar{M}=\frac{m+M+(K_g*J_m)}{r^2}$\\
$C_1=\bar{B}\dot{x}-\frac{1}{2}ml\sin\left(\theta\right)\dot{\theta}^2$\\
$C_2=-\frac{1}{2}mgl\sin\left(\theta\right)$\\
$\bar{B}=\frac{K_gB_m}{r^2}+\frac{K_g^2K_iK_b}{r^2R_a}$\\
$B_l=\frac{K_gK_i}{rR_a}$
\end{itemize}


\end{frame}

\begin{frame}{Linearized System Model}
\begin{itemize}
\pause \item $\dot{X}=AX+Bu$ \\ \pause $X=\left[ \begin{array}{c} x_1 \\ x_{2} \\ x_3 \\ x_{4} \end{array} \right]=\left[ \begin{array}{cccc} x-x_s \\ \dot{x} \\ \theta \\ \dot{\theta} \end{array} \right]$, \pause $A=\left[ \begin{array}{cccc} 0 & 1 & 0 & 0 \\ 0 & -a_{22} & -a_{23} & a_{24} \\ 0 & 0 & 0 & 1 \\ 0 & a_{42} & a_{43} & -a_{44} \end{array} \right]$, \pause $B=\left[ \begin{array}{c} 0 \\ b_{2} \\ 0 \\ -b_{4} \end{array} \right]$
\pause \item Notable linearizations are $\sin\left(\theta\right)\approx\theta$ for small $\theta$ and armature inductance $L_a=0$ to remove the armature current $I_a$ state variable, reducing the order of the system, thus leaving only armature voltage $V_a$ for control ($u=V_a$)
\end{itemize}

\end{frame}

\begin{frame}{Linear State Feedback}

\begin{itemize}
\pause \item Linear model \\$\dot{X}\left(t\right)=AX\left(t\right)+Bu\left(t\right)$
\pause \item Variation-of-constants formula (linear solution) \\ $X\left(t\right)=e^{A\left(t-t_0\right)}x\left(t_0\right)+\int_{t_0}^{t}e^{A\left(t-\tau\right)}Bu\left(\tau\right)d\tau$
\pause \item Control is of the form \\$u\left(t\right)=KX\left(t\right)$
\pause \item Linear state feedback \\ $\dot{X}\left(t\right)=\left(A+BK\right)X\left(t\right)$
\pause \item Linear state feedback solution \\ $X\left(t\right)=e^{\left(A+BK\right)\left(t-t_0\right)}X\left(t_0\right)$
\pause \item First-order approximations for $\dot{x}$ and $\dot{\theta}$\\
$\dot{\theta}(t)=\frac{[\theta(t)-\theta(t-mT_s)]}{mT_s}$\\
$\dot{x}(t)=\frac{[x(t)-x(t-mT_s)]}{mT_s}$\\
where $m$ is an integer greater than one (chosen as $2$ by experimentation)
\end{itemize}

\end{frame}

%\begin{frame}{System Model}
%
%\begin{center}
%\begin{figure}[h!]
%	\begin{tikzpicture}[auto, node distance=2cm,>=latex']
%			\tikzstyle{block} = [draw, fill=blue!20, rectangle, 
%			    minimum height=3em, minimum width=6em]
%			\tikzstyle{sum} = [draw, fill=blue!20, circle, node distance=1cm]
%			\tikzstyle{input} = [coordinate]
%			\tikzstyle{output} = [coordinate]
%			\tikzstyle{pinstyle} = [pin edge={to-,thin,black}]
%			
%	    % We start by placing the blocks
%	    \node [input, name=topleft] {};
%	    \node [block, below of=topleft] (actuator) {Actuator};
%	    \node [input, name=botleft, below of=actuator] {};
%	    \node [block, right of=topleft] (plant) {Plant};
%			\node [block, right of=botleft] (controller) {Controller};
%			\node [input, name=center, below of=plant] {};
%			\node [block, right of=center] (sensor) {Sensor};
%			\node [input, name=topright, above of=sensor] {};
%			\node [input, name=botright, below of=sensor] {};
%	
%			\draw [->] (topleft) -- (plant);
%			\draw [-] (plant) -- (topright);
%			\draw [->] (topright) -- node {$x\left(t\right)$, $\theta\left(t\right)$} (sensor);
%			\draw [-] (sensor) -- node {$x\left(t-\tau\right)$, \\$\theta\left(t-\tau\right)$} (botright);
%			\draw [->] (botright) -- (controller);
%			\draw [-] (controller) -- (botleft);
%			\draw [->] (botleft) -- node {$kX\left(t-\tau\right)$} (actuator);
%			\draw [-] (actuator) -- node {$kX\left(t-\tau-\tau_a\right)$} (topleft);
%	
%	    % Once the nodes are placed, connecting them is easy. 
%	%    \draw [draw,->] (input) -- node {$r$} (sum);
%	%    \draw [->] (sum) -- node {$e$} (controller);
%	%    \draw [->] (system) -- node [name=y] {$y$}(output);
%	%    \draw [->] (y) |- (measurements);
%	%    \draw [->] (measurements) -| node[pos=0.99] {$-$} node [near end] {$y_m$} (sum);
%	\end{tikzpicture}
%	\caption{System Model}
%	\label{fig:systemModel}
%\end{figure}
%\end{center}
%
%
%\end{frame}

\begin{frame}{System Constraints}

\begin{itemize}
\pause \item Plant Constraints\\
\pause $x\in\left[-0.7,0.7\right]$ meters\\
\pause $\dot{x}\in\left[-1.0,1.0\right]$ meters/second\\
\pause $\theta\in\left[-30,30\right]^{\circ}$\\
\pause $\dot{\theta}$ is unconstrained
%\begin{equation}
%X=\left[ \begin{array}{c} x_1 \\ x_{2} \\ x_3 \\ x_{4} \end{array} \right]=\left[ \begin{array}{cccc} x-x_s \\ \dot{x} \\ \theta \\ \dot{\theta} \end{array} \right]
%\label{eq:stateMatrix}
%\end{equation}
\pause \item Control Constraints\\
\pause $V_a\in\left[-4.96,4.96\right]$ volts
\pause \item Feasible Region\\
Defined by these constraints, that is, only certain physical conditions are feasible for the system
\pause \item Stabilizable Region\\
This is a subset of the feasible region, determined by solving a linear-matrix inequality (LMI) problem to find controller gains that maximize the region\\

\end{itemize}

\end{frame}

\begin{frame}{Simplex Architecture}
\begin{itemize}
\pause \item Analytical Redundancy
\pause \item Controllers
\begin{itemize}
\pause \item Safety Controller ($u_{sc}\left(t\right)=K_{sc}X\left(t\right)$)
\pause \item Baseline Controller ($u_{bc}\left(t\right)=K_{bc}X\left(t\right)$)
\pause \item Experimental Controller ($u_{ec}\left(t\right)=K_{ec}X\left(t\right)$)
\end{itemize}
\pause \item Switching Controller\\Choose $u_{\sigma}\in\left[u_{sc}, u_{bc}, u_{ec}\right]$ based on the current (and future) stabilizable region(s) the system is within
\pause \item Entire system modeled using HIOA (not shown here due to time-constraints)
\end{itemize}

\end{frame}

\begin{frame}{Simplex Architecture}

\scalebox{0.9} % Change this value to rescale the drawing.
{
\begin{pspicture}(0,-4.2592187)(15.362812,4.2992187)
\definecolor{color6b}{rgb}{1.0,1.0,0.8}
\definecolor{color12b}{rgb}{1.0,0.6,0.6}
\definecolor{color0b}{rgb}{0.8,0.8,1.0}
\definecolor{color169b}{rgb}{0.8,1.0,0.8}
\definecolor{color3b}{rgb}{0.8,1.0,0.6}
\pause
\psframe[linewidth=0.03,dimen=outer,fillstyle=solid,fillcolor=color6b](11.180938,2.5807812)(8.220938,1.1407813)
\psframe[linewidth=0.03,dimen=outer,fillstyle=solid,fillcolor=color12b](3.2609375,2.5407813)(0.3009375,1.1007812)
\psframe[linewidth=0.03,dimen=outer,fillstyle=solid,fillcolor=color0b](7.0009375,3.9407814)(4.0409374,2.5007813)
\psframe[linewidth=0.04,dimen=outer,fillstyle=solid,fillcolor=color169b](7.8209376,1.9607812)(3.6209376,-4.2592187)
\usefont{T1}{ptm}{m}{n}
\rput(5.588281,3.0957813){\large $\dot{X}=f\left(X,u\right)$}
\usefont{T1}{ptm}{m}{n}
\rput(5.39875,3.5757813){\large Plant}
\usefont{T1}{ptm}{m}{n}
\rput(9.740781,2.2107813){Sensor}
\usefont{T1}{ptm}{m}{n}
\rput(10.532344,0.19078125){$x\left(t-\tau\right)$}
\usefont{T1}{ptm}{m}{n}
\rput(10.492344,-0.16921875){$\theta\left(t-\tau\right)$}
\usefont{T1}{ptm}{m}{n}
\rput(9.790313,1.7907813){$\tau$ delay}
\usefont{T1}{ptm}{m}{n}
\rput(5.860781,1.6707813){Switching Controller}
\psline[linewidth=0.04,arrowsize=0.05291667cm 2.0,arrowlength=1.4,arrowinset=0.4]{->}(9.760938,1.1207813)(9.760938,-1.8592187)(7.9009376,-1.8392187)(7.8209376,-1.8392187)
\psline[linewidth=0.04,arrowsize=0.05291667cm 2.0,arrowlength=1.4,arrowinset=0.4]{->}(3.6409376,-1.8592187)(1.6409374,-1.8792187)(1.6809375,1.0807812)
\psline[linewidth=0.04,arrowsize=0.05291667cm 2.0,arrowlength=1.4,arrowinset=0.4]{->}(1.6809375,2.5407813)(1.6809375,3.5407813)(4.0609374,3.5207813)
\psline[linewidth=0.04,arrowsize=0.05291667cm 2.0,arrowlength=1.4,arrowinset=0.4]{->}(7.0009375,3.5607812)(9.800938,3.5207813)(9.840938,2.5407813)
\usefont{T1}{ptm}{m}{n}
\rput(1.640625,2.2507813){Actuator}
\usefont{T1}{ptm}{m}{n}
\rput(8.862344,3.7707813){$\left[x\left(t\right) \dot{x}\left(t\right) \theta\left(t\right) \dot{\theta}\left(t\right)\right]^T$}
\usefont{T1}{ptm}{m}{n}
\rput(2.5523438,-0.00921875){$u_{\sigma}\left(t-\tau\right)$}
\usefont{T1}{ptm}{m}{n}
\rput(2.6123438,3.7707813){$u_{\sigma}\left(t-\tau-\tau_a\right)$}
\usefont{T1}{ptm}{m}{n}
\rput(1.6803125,1.7907813){$\tau_a$ delay}
\usefont{T1}{ptm}{m}{n}
\rput(8.912344,4.110781){$X\left(t\right)=$}
\pause
\psframe[linewidth=0.03,dimen=outer,fillstyle=solid,fillcolor=color3b](7.2809377,1.1807812)(4.3209376,-0.25921875)
\usefont{T1}{ptm}{m}{n}
\rput(5.920781,0.87078124){Safety Controller}
\usefont{T1}{ptm}{m}{n}
\rput(5.802344,0.49078125){$u_{sc}\left(t-\tau\right)=$}
\usefont{T1}{ptm}{m}{n}
\rput(5.6723437,0.09078125){$k_{sc}X\left(t-\tau\right)$}
\pause
\psframe[linewidth=0.03,dimen=outer,fillstyle=solid,fillcolor=color3b](7.2809377,-0.7392188)(4.3209376,-2.1792188)
\usefont{T1}{ptm}{m}{n}
\rput(5.8101563,-0.9892188){Baseline Ctrl.}
\usefont{T1}{ptm}{m}{n}
\rput(5.722344,-1.3492187){$u_{bc}\left(t-\tau\right)=$}
\usefont{T1}{ptm}{m}{n}
\rput(5.6523438,-1.7892188){$k_{bc}X\left(t-\tau\right)$}
\pause
\psframe[linewidth=0.03,dimen=outer,fillstyle=solid,fillcolor=color3b](7.2609377,-2.6392188)(4.3009377,-4.079219)
\usefont{T1}{ptm}{m}{n}
\rput(5.789375,-2.8892188){Experimental Ctrl.}
\usefont{T1}{ptm}{m}{n}
\rput(5.7923436,-3.2492187){$u_{ec}\left(t-\tau\right)=$}
\usefont{T1}{ptm}{m}{n}
\rput(5.702344,-3.6492188){$k_{ec}X\left(t-\tau\right)$}
\end{pspicture}
}


\end{frame}

\begin{frame}{Small-Gain Theorem}

\begin{itemize}
\pause \item If the original system $\dot{X}=f\left(X,u\right)$ is input-to-state stable (ISS), and a disturbance input is small enough, then the system with disturbance is ISS.
\pause \item What is small enough?  \pause Effectively less than a system's stability margin, in the LTI case, the real-axis distance of eigenvalues to the right-half plane.
\pause \item In the case of a disturbance (error) caused by delay, this allows one to bound 	the delay by bounding the error caused by the delay, allowing one to determine a tolerable (not maximal) delay to ensure ISS.
\end{itemize}

\end{frame}

\begin{frame}{Small-Gain Theorem - Coupling with Disturbance from Delay}

\pause

% Generated with LaTeXDraw 2.0.1
% Tue Dec 02 17:16:33 CST 2008
% \usepackage[usenames,dvipsnames]{pstricks}
% \usepackage{epsfig}
% \usepackage{pst-grad} % For gradients
% \usepackage{pst-plot} % For axes
\scalebox{1} % Change this value to rescale the drawing.
{
\begin{pspicture}(0,-2.4)(9.522813,2.4)
\definecolor{color0b}{rgb}{0.8,0.8,1.0}
\psframe[linewidth=0.03,dimen=outer,fillstyle=solid,fillcolor=color0b](6.9609375,2.4)(2.3609376,0.58)
\psframe[linewidth=0.03,dimen=outer,fillstyle=solid,fillcolor=color0b](6.9809375,-0.58)(2.3809376,-2.4)
\psline[linewidth=0.04,arrowsize=0.05291667cm 2.0,arrowlength=1.4,arrowinset=0.4]{->}(6.9609375,1.62)(8.180938,1.6)(8.180938,-1.6)(6.9809375,-1.6)
\psline[linewidth=0.04,arrowsize=0.05291667cm 2.0,arrowlength=1.4,arrowinset=0.4]{->}(2.3609376,-1.6)(1.3409375,-1.62)(1.3809375,1.66)(2.4009376,1.64)
\usefont{T1}{ptm}{m}{n}
\rput(4.722344,1.67){$\dot{X}\left(t\right)=f\left(X\left(t\right),u\left(t-\tau\right)\right)$}
\usefont{T1}{ptm}{m}{n}
\rput(4.4026564,-1){Disturbance}
\usefont{T1}{ptm}{m}{n}
\rput(4.602344,-1.5){$\epsilon\left(t\right)=KX\left(t-\tau\right)-KX\left(t\right)$}
\end{pspicture} 
}

\end{frame}

\begin{frame}{Bounding Our System for ISS and Small-Gains}

\begin{itemize}

\item Delayed control $u\left(t\right)=KX\left(t-\tau\right)$ \pause$=KX\left(t\right)+\epsilon\left(t\right)$
\pause \item Error from delay \\ $\epsilon\left(t\right)=KX\left(t-\tau\right)-KX\left(t\right)$
\pause \item System model with disturbance from delay \\ $\dot{X}\left(t\right)=\left(A+BK\right)X\left(t\right)+B\epsilon\left(t\right)$
\pause \item Small Gain Argument\\$\epsilon=-\int^{t}_{t-\tau}K\left(AX\left(s\right)+BKX\left(s-\tau\right)\right)ds$
\pause \\ $|\epsilon|\leq\tau\left(||KA||+||KBK||\right)\cdot||X||_{\left[t-2\tau,t\right]}$
\pause \\ $d=(||KA||+||KBK||)$
\pause \item Lyapunov function satisfied by $\dot{V}=-X^TQX+X^TPB\epsilon$
\pause \\ Conservative bound \\ $\dot{V}\leq-X^TQX+|X|\cdot|\epsilon|\cdot||PB||$
\pause \\ $\dot{V}\leq-\lambda_{min}\left(Q\right)|X|^2+|X|\cdot|\epsilon|\cdot||PB||$
\pause \\ $\dot{V}=-|X|\left(\lambda_{min}\left(Q\right)|X|-||PB||\cdot|\epsilon|\right)$
\pause \\ $|x|>\frac{||PB||}{\lambda_{min}\left(Q\right)}\cdot|\epsilon|$
\pause \\ Define $\rho\left(r\right)=cr$ where $c=\frac{||PB||}{\lambda_{min}\left(Q\right)}$
\end{itemize}

\end{frame}

\begin{frame}{Small-Gain Theorem for Our Case}

\begin{itemize}
\item Assume the undelayed (undisturbed) system is ISS, then if there exists some $\tau$ such that $\tau cd<1$, \pause the disturbed system is ISS with respect to the disturbance caused by the delay error ($\epsilon$), and a tolerable delay is $\tau<\frac{1}{cd}$.
\pause \item For this system, the following tolerable delays were found for each controller $\tau_{sc}=9$ms, $\tau_{bc}=12.3$ms, $\tau_{ec}=18.6$ms.
\end{itemize}

\end{frame}

\begin{frame}{Conclusions and Future Work}

\begin{itemize}
\pause \item Control period is $T_s=20$ms, so this method cannot guarantee stability with respect to the actual delay experienced, so model-based state projection is necessary.
\pause \item Note that small-gain is only a sufficient condition, so it could be the case that it is stable without state projection.  A future line of work could be to define and solve an optimization problem to find the maximum tolerable $\tau$.
\pause \item When perfect state projection is considered, stability is then guaranteed for a delay $T_s+\tau$.  State projection is not perfect here however, due primarily to the first-order approximation used to reconstruct $\dot{x}$ and $\dot{\theta}$ since they are unobservable, in addition to other errors like quantization not explicitly shown here.  Further work could be done here as well.
\end{itemize}

\end{frame}

\begin{frame}{Questions}
\small ? \Large ? \Huge ? \Giant ? \CrazyBig ?
\end{frame}

\begin{frame}{Questions: Lyapunov Stability}

\begin{itemize}
\pause \item Stable in the Sense of Lyapunov (isL) if\\
$\forall \epsilon>0 .\exists \delta=\delta\left(\epsilon\right)>0$ such that, if $||x\left(0\right)||<\delta$, then $||x\left(t\right)||<\epsilon$, $\forall t\geq0$
\pause \item Asymptotically Stable\\If isL and if $\exists \delta > 0$ such that if $||x\left(0\right)||<\delta$, then $x(t)\rightarrow0$ as $t\rightarrow\inf$
\pause \item LTI Case\\If $\exists P=P^T>0$ for some $Q=Q^T>0$ that satisfies Lyapunov equation $A^TP+PA+Q=0$, then the system is asymptotically stable, and the Lyapunov function is $V=X^TPX$
\end{itemize}

\end{frame}

\begin{frame}{Questions: Input-to-State Stability (ISS)}
\begin{itemize}
\pause \item Comparison Functions of Hahn\\
Class $\K$: continuous, strictly increasing\\
Class $\K_\infty$: same as class $\K$ and unbounded\\
Class $\KL$: continuous, strictly decreasing
\pause \item Consider $\dot{x}\left(t\right)=f\left(x\left(t\right),w\left(t\right)\right)$ with solutions $\varphi\left(t,x,w\right)$
\pause \item System is ISS if $\exists\beta\in\KL$ and $\gamma\in\K_\infty$ such that $\forall x\left(t_0\right)$, $\forall w$, and $\forall t\geq0$, then $||\varphi\left(t,x,w\right)||\leq\max\left\{\beta\left(||x||,t\right),\gamma\left(||w||_\infty\right)\right\}$
\end{itemize}
\end{frame}

\begin{frame}{Questions: Nonlinear System Model Derived from Energy}

\begin{itemize}
\item Coordinates of Small Portion of Bar with Mass $dm$\\
$x_{dm}=x+q\sin\left(\theta\right)\Rightarrow\dot{x}_{dm}=\dot{x}+q\cos\left(\theta\right)\dot{\theta}$\\$y_{dm}=q\cos\left(\theta\right)\Rightarrow\dot{y}_{dm}=-q\sin\left(\theta\right)\dot{\theta}$
\pause \item Kinetic Energy of Small Portion of Bar with Mass $dm$\\$K_{dm}=\frac{1}{2}dm\left(\dot{x}_{dm}^2+\dot{y}_{dm}^2\right)$ \pause $=\frac{1}{2}\left(dm\right)\left(\dot{x}^2+2q\cos\left(\theta\right)\dot{x}\dot{\theta}+q^2\dot{\theta}^2\right)$
\pause \item Potential Energy of Small Portion of Bar with Mass $dm$\\$P_{dm}=\left(dm\right)g\cos\left(\theta\right)$
\pause \item Integrating over Bar Length $0$ to $l$\\
$K=K_c+K_p=\frac{1}{2}\left(M+m\right)\dot{x}^2+\frac{1}{2}ml\cos\left(\theta\right)\dot{x}\dot{\theta}+\frac{1}{6}ml^2\dot{\theta}^2$\\
$P=\frac{1}{2}mgl\cos\left(\theta\right)$
\pause \item Lagrangian $L=K-P$
\pause \item Euler-Lagrange Equations for Forces on System\\
$\frac{d}{dt}\frac{\partial L}{\partial\dot{x}}-\frac{\partial L}{\partial x}=F-f_c$ and $\frac{d}{dt}\frac{\partial L}{\partial\dot{\theta}}-\frac{\partial L}{\partial \theta}=-f_p$
\end{itemize}

\end{frame}

\begin{frame}{Questions: Plant HIOA}

\begin{figure}[h!]
\centering
  \hrule
%  {\lstinputlisting[language=ioa,lastline=8]{Plant.hioa}}
%  {\lstinputlisting[language=ioa,firstline=9]{Plant.hioa}}
  {\lstinputlisting[language=ioa,firstline=1]{Plant.hioa}}
  \hrule
  \caption{Linearized Plant Model}
  \label{fig:plant}
\end{figure}

\end{frame}

\begin{frame}{Questions: Sensor HIOA}

\begin{figure}[h!]
\centering
  \hrule
%  \two{.47}{.47}
%  {\lstinputlisting[language=ioa,lastline=11]{Sensor.hioa}}
%  {\lstinputlisting[language=ioa,firstline=12]{Sensor.hioa}}
	{\lstinputlisting[language=ioa,firstline=1]{Sensor.hioa}}
  \hrule
  \caption{Sensor}
  \label{fig:sensor}
\end{figure}

\end{frame}

\begin{frame}{Questions: Actuator HIOA}
\begin{figure}[h!]
\centering
  \hrule
%  \two{.47}{.47}
%  {\lstinputlisting[language=ioa,lastline=14]{Actuator.hioa}}
%  {\lstinputlisting[language=ioa,firstline=15]{Actuator.hioa}}
	{\lstinputlisting[language=ioa,firstline=1]{Actuator.hioa}}
  \hrule
  \caption{Actuator}
  \label{fig:actuator}
\end{figure}
\end{frame}

\begin{frame}{Questions: Safety, Baseline, Experimental Controller HIOAs}

\begin{figure}[h!]
\centering
  \hrule
%  \two{.47}{.47}
  {\lstinputlisting[language=ioa,lastline=15]{SafetyController.hioa}}
%  {\lstinputlisting[language=ioa,firstline=15]{SafetyController.hioa}}
%  {\lstinputlisting[language=ioa,firstline=1]{SafetyController.hioa}}
  \hrule
  \caption{Safety Controller}
  \label{fig:safetyController}
\end{figure}

\end{frame}

\begin{frame}{Questions: Safety, Baseline, Experimental Controller HIOAs (cont)}

\begin{figure}[h!]
\centering
  \hrule
%  \two{.47}{.47}
%  {\lstinputlisting[language=ioa,lastline=15]{SafetyController.hioa}}
  {\lstinputlisting[language=ioa,firstline=16]{SafetyController.hioa}}
%  {\lstinputlisting[language=ioa,firstline=1]{SafetyController.hioa}}
  \hrule
  \caption{Safety Controller}
  \label{fig:safetyController}
\end{figure}

\end{frame}


\end{document}
